% Options for packages loaded elsewhere
\PassOptionsToPackage{unicode}{hyperref}
\PassOptionsToPackage{hyphens}{url}
%
\documentclass[
  man,floatsintext]{apa6}
\usepackage{amsmath,amssymb}
\usepackage{lmodern}
\usepackage{iftex}
\ifPDFTeX
  \usepackage[T1]{fontenc}
  \usepackage[utf8]{inputenc}
  \usepackage{textcomp} % provide euro and other symbols
\else % if luatex or xetex
  \usepackage{unicode-math}
  \defaultfontfeatures{Scale=MatchLowercase}
  \defaultfontfeatures[\rmfamily]{Ligatures=TeX,Scale=1}
\fi
% Use upquote if available, for straight quotes in verbatim environments
\IfFileExists{upquote.sty}{\usepackage{upquote}}{}
\IfFileExists{microtype.sty}{% use microtype if available
  \usepackage[]{microtype}
  \UseMicrotypeSet[protrusion]{basicmath} % disable protrusion for tt fonts
}{}
\makeatletter
\@ifundefined{KOMAClassName}{% if non-KOMA class
  \IfFileExists{parskip.sty}{%
    \usepackage{parskip}
  }{% else
    \setlength{\parindent}{0pt}
    \setlength{\parskip}{6pt plus 2pt minus 1pt}}
}{% if KOMA class
  \KOMAoptions{parskip=half}}
\makeatother
\usepackage{xcolor}
\usepackage{graphicx}
\makeatletter
\def\maxwidth{\ifdim\Gin@nat@width>\linewidth\linewidth\else\Gin@nat@width\fi}
\def\maxheight{\ifdim\Gin@nat@height>\textheight\textheight\else\Gin@nat@height\fi}
\makeatother
% Scale images if necessary, so that they will not overflow the page
% margins by default, and it is still possible to overwrite the defaults
% using explicit options in \includegraphics[width, height, ...]{}
\setkeys{Gin}{width=\maxwidth,height=\maxheight,keepaspectratio}
% Set default figure placement to htbp
\makeatletter
\def\fps@figure{htbp}
\makeatother
\setlength{\emergencystretch}{3em} % prevent overfull lines
\providecommand{\tightlist}{%
  \setlength{\itemsep}{0pt}\setlength{\parskip}{0pt}}
\setcounter{secnumdepth}{-\maxdimen} % remove section numbering
% Make \paragraph and \subparagraph free-standing
\ifx\paragraph\undefined\else
  \let\oldparagraph\paragraph
  \renewcommand{\paragraph}[1]{\oldparagraph{#1}\mbox{}}
\fi
\ifx\subparagraph\undefined\else
  \let\oldsubparagraph\subparagraph
  \renewcommand{\subparagraph}[1]{\oldsubparagraph{#1}\mbox{}}
\fi
\newlength{\cslhangindent}
\setlength{\cslhangindent}{1.5em}
\newlength{\csllabelwidth}
\setlength{\csllabelwidth}{3em}
\newlength{\cslentryspacingunit} % times entry-spacing
\setlength{\cslentryspacingunit}{\parskip}
\newenvironment{CSLReferences}[2] % #1 hanging-ident, #2 entry spacing
 {% don't indent paragraphs
  \setlength{\parindent}{0pt}
  % turn on hanging indent if param 1 is 1
  \ifodd #1
  \let\oldpar\par
  \def\par{\hangindent=\cslhangindent\oldpar}
  \fi
  % set entry spacing
  \setlength{\parskip}{#2\cslentryspacingunit}
 }%
 {}
\usepackage{calc}
\newcommand{\CSLBlock}[1]{#1\hfill\break}
\newcommand{\CSLLeftMargin}[1]{\parbox[t]{\csllabelwidth}{#1}}
\newcommand{\CSLRightInline}[1]{\parbox[t]{\linewidth - \csllabelwidth}{#1}\break}
\newcommand{\CSLIndent}[1]{\hspace{\cslhangindent}#1}
\ifLuaTeX
\usepackage[bidi=basic]{babel}
\else
\usepackage[bidi=default]{babel}
\fi
\babelprovide[main,import]{english}
% get rid of language-specific shorthands (see #6817):
\let\LanguageShortHands\languageshorthands
\def\languageshorthands#1{}
% Manuscript styling
\usepackage{upgreek}
\captionsetup{font=singlespacing,justification=justified}

% Table formatting
\usepackage{longtable}
\usepackage{lscape}
% \usepackage[counterclockwise]{rotating}   % Landscape page setup for large tables
\usepackage{multirow}		% Table styling
\usepackage{tabularx}		% Control Column width
\usepackage[flushleft]{threeparttable}	% Allows for three part tables with a specified notes section
\usepackage{threeparttablex}            % Lets threeparttable work with longtable

% Create new environments so endfloat can handle them
% \newenvironment{ltable}
%   {\begin{landscape}\centering\begin{threeparttable}}
%   {\end{threeparttable}\end{landscape}}
\newenvironment{lltable}{\begin{landscape}\centering\begin{ThreePartTable}}{\end{ThreePartTable}\end{landscape}}

% Enables adjusting longtable caption width to table width
% Solution found at http://golatex.de/longtable-mit-caption-so-breit-wie-die-tabelle-t15767.html
\makeatletter
\newcommand\LastLTentrywidth{1em}
\newlength\longtablewidth
\setlength{\longtablewidth}{1in}
\newcommand{\getlongtablewidth}{\begingroup \ifcsname LT@\roman{LT@tables}\endcsname \global\longtablewidth=0pt \renewcommand{\LT@entry}[2]{\global\advance\longtablewidth by ##2\relax\gdef\LastLTentrywidth{##2}}\@nameuse{LT@\roman{LT@tables}} \fi \endgroup}

% \setlength{\parindent}{0.5in}
% \setlength{\parskip}{0pt plus 0pt minus 0pt}

% Overwrite redefinition of paragraph and subparagraph by the default LaTeX template
% See https://github.com/crsh/papaja/issues/292
\makeatletter
\renewcommand{\paragraph}{\@startsection{paragraph}{4}{\parindent}%
  {0\baselineskip \@plus 0.2ex \@minus 0.2ex}%
  {-1em}%
  {\normalfont\normalsize\bfseries\itshape\typesectitle}}

\renewcommand{\subparagraph}[1]{\@startsection{subparagraph}{5}{1em}%
  {0\baselineskip \@plus 0.2ex \@minus 0.2ex}%
  {-\z@\relax}%
  {\normalfont\normalsize\itshape\hspace{\parindent}{#1}\textit{\addperi}}{\relax}}
\makeatother

% \usepackage{etoolbox}
\makeatletter
\patchcmd{\HyOrg@maketitle}
  {\section{\normalfont\normalsize\abstractname}}
  {\section*{\normalfont\normalsize\abstractname}}
  {}{\typeout{Failed to patch abstract.}}
\patchcmd{\HyOrg@maketitle}
  {\section{\protect\normalfont{\@title}}}
  {\section*{\protect\normalfont{\@title}}}
  {}{\typeout{Failed to patch title.}}
\makeatother

\usepackage{xpatch}
\makeatletter
\xapptocmd\appendix
  {\xapptocmd\section
    {\addcontentsline{toc}{section}{\appendixname\ifoneappendix\else~\theappendix\fi\\: #1}}
    {}{\InnerPatchFailed}%
  }
{}{\PatchFailed}
\keywords{keywords\newline\indent Word count: X}
\usepackage{lineno}

\linenumbers
\usepackage{csquotes}
\raggedbottom
\usepackage{float}
\ifLuaTeX
  \usepackage{selnolig}  % disable illegal ligatures
\fi
\IfFileExists{bookmark.sty}{\usepackage{bookmark}}{\usepackage{hyperref}}
\IfFileExists{xurl.sty}{\usepackage{xurl}}{} % add URL line breaks if available
\urlstyle{same} % disable monospaced font for URLs
\hypersetup{
  pdftitle={Supplement: Moral Dilution},
  pdfauthor={Cillian McHugh1 \& Eric R. Igou1},
  pdflang={en-EN},
  pdfkeywords={keywords},
  hidelinks,
  pdfcreator={LaTeX via pandoc}}

\title{Supplement: Moral Dilution}
\author{Cillian McHugh\textsuperscript{1} \& Eric R. Igou\textsuperscript{1}}
\date{}


\shorttitle{Moral Dilution}

\authornote{

Department of Psychology, University of Limerick.

All procedures performed in studies involving human participants were approved by institutional research ethics committee and conducted in accordance with the Code of Professional Ethics of the Psychological Society of Ireland, and with the 1964 Helsinki declaration and its later amendments or comparable ethical standards. Informed consent was obtained from all individual participants included in the study. The authors declare that there are no potential conflicts of interest with respect to the research, authorship, and/or publication of this article. All authors consented to the submission of this manuscript.

Correspondence concerning this article should be addressed to Cillian McHugh, University of Limerick, Limerick, Ireland, V94 T9PX. E-mail: \href{mailto:cillian.mchugh@ul.ie}{\nolinkurl{cillian.mchugh@ul.ie}}

}

\affiliation{\vspace{0.5cm}\textsuperscript{1} University of Limerick}

\abstract{%
Supplementary analysis to accompany the manuscript The Moral Dilution Effect: Irrelevant Information Influences Judgments of Moral Character.
}



\begin{document}
\maketitle

\hypertarget{pilot-study-1}{%
\section{Pilot Study 1}\label{pilot-study-1}}

The aim of this pilot study was to develop and test materials that could be used to study the dilution effect for moral characters. We developed diagnostic and non-diagnostic character descriptions. We hypothesized that moral evaluations of the diagnostic descriptions would be more severe (more immoral) than for the non-diagnostic descriptions.

\hypertarget{pilot-study-1-method}{%
\subsection{Pilot Study 1: Method}\label{pilot-study-1-method}}

\hypertarget{pilot-1-participants-and-design}{%
\subsubsection{Pilot 1: Participants and design}\label{pilot-1-participants-and-design}}

The pilot study was a within-subjects design. The independent variable was description type with two levels, \emph{diagnostic} and \emph{non-diagnostic}. We used two dependent variables. The first dependent variable was the four item moral perception scale (MPS-4), participants rated the characters on four dimensions using 7-point bipolar scales. The dimensions and scale endpoints were: Bad-Good, Immoral-Moral, Violent-Peaceful, Merciless-Empathetic, this showed excellent reliability, \(\alpha\) = 0.93. The second dependent variable was a single item moral perception measure (MM-1) which consisted of a 100-point slider ranging from 0 = \emph{Very Immoral} to 100 = \emph{Very Moral}. Both dependent variables were taken from Walker et al. (2021).

A total sample of 235 (89 female, 142 male, 1 non-binary, 1 prefer not to say; \emph{M}\textsubscript{age} = 36.45, min = 20, max = 72, \emph{SD} = 10.23) started the survey. Participants were recruited from MTurk.

We removed participants who failed both manipulation checks (\emph{n} = 23), leaving a total sample of 212 participants (80 female, 128 male, 1 non-binary, 1 prefer not to say; \emph{M}\textsubscript{age} = 36.63, min = 20, max = 72, \emph{SD} = 10.34).

\hypertarget{pilot-1-procedure-and-materials}{%
\subsubsection{Pilot 1: Procedure and materials}\label{pilot-1-procedure-and-materials}}

Data were collected using an online questionnaire presented with Qualtrics (www.qualtrics.com). Participants were presented with descriptions of six characters.

Moral character descriptions were developed by combining descriptions relating to three different moral foundations. These descriptions were adapted from the items of the extended character morality questionnaire (Grizzard et al., 2020), and read as follows: (i) \emph{Imagine a person named Sam. Throughout their life they have been known to be cruel, act unfairly, and to betray their own group}; (ii) \emph{Imagine a person named Robin. Throughout their life they have been known to physically hurt others, treat some people differently to others, and show lack of loyalty}; (iii) \emph{Imagine a person named Francis. Throughout their life they have been known to violate the standards of purity and decency, show lack of respect for authority, and treat people unequally} (iv) \emph{Imagine a person named Alex. Throughout their life they have been known to cause others to suffer emotionally, to deny others their rights, and to cause chaos or disorder}.

We developed neutral descriptions that included information relating to physical appearance/attributes, hobbies/activities, and family information that read as follows: (i) \emph{Imagine a person named Jackie. They have red hair, play tennis four times a month, and have one older sibling and one younger sibling}; (ii) \emph{Imagine a person named Charlie. They are left-handed, drink tea in the morning, and have two older siblings and one younger sibling}.

Character descriptions did not specify the gender of the charcters, and all characters had names that could be either male or female (Sam, Robin, Francis, Alex, Jackie, Charlie). All participants read six descriptions, four moral descriptions and two neutral. Pilot Study 1 was pre-registered at \color{blue}\url{https://aspredicted.org/3VK_8FD}\color{black}.

\hypertarget{pilot-1-results}{%
\subsection{Pilot 1: Results}\label{pilot-1-results}}

The means and standard deviations for MPS-4 for each scenario are as follows:
\emph{Sam} (diagnostic),
\emph{M}\textsubscript{MPS-4} = 4.35, \emph{SD}\textsubscript{MPS-4} = 1.90,
\emph{Francis} (diagnostic),
\emph{M}\textsubscript{MPS-4} = 4.46, \emph{SD}\textsubscript{MPS-4} = 1.73,
\emph{Alex} (diagnostic),
\emph{M}\textsubscript{MPS-4} = 4.44, \emph{SD}\textsubscript{MPS-4} = 1.79,
\emph{Robin} (diagnostic),
\emph{M}\textsubscript{MPS-4} = 4.35, \emph{SD}\textsubscript{MPS-4} = 1.96,
\emph{Jackie} (non-diagnostic),
\emph{M}\textsubscript{MPS-4} = 5.40, \emph{SD}\textsubscript{MPS-4} = 1.01,
\emph{Charlie} (non-diagnostic),
\emph{M}\textsubscript{MPS-4} = 5.38, \emph{SD}\textsubscript{MPS-4} = 1.01. For the diagnostic descriptions, there was no significant variation depending on the description, \emph{F}(3,600) = 1.58, \emph{p} = .194, partial \(\eta\)\textsuperscript{2} = 0.00. For the non-diagnostic descriptions there was no significant difference in ratings depending on description, \emph{t}(211) = -0.67, \emph{p} = .506, \emph{d} = 0.05.

The means and standard deviations for MM-1 for each scenario are as follows:
\emph{Sam} (diagnostic),
\emph{M}\textsubscript{MM-1} = 55.67, \emph{SD}\textsubscript{MM-1} = 30.47;
\emph{Francis} (diagnostic),
\emph{M}\textsubscript{MM-1} = 58.22, \emph{SD}\textsubscript{MM-1} = 28.61;
\emph{Alex} (diagnostic),
\emph{M}\textsubscript{MM-1} = 56.80, \emph{SD}\textsubscript{MM-1} = 29.45;
\emph{Robin} (diagnostic),
\emph{M}\textsubscript{MM-1} = 55.49, \emph{SD}\textsubscript{MM-1} = 31.38;
\emph{Jackie} (non-diagnostic),
\emph{M}\textsubscript{MM-1} = 73.00, \emph{SD}\textsubscript{MM-1} = 14.72;
\emph{Charlie} (non-diagnostic),
\emph{M}\textsubscript{MM-1} = 72.94, \emph{SD}\textsubscript{MM-1} = 14.79. For the diagnostic descriptions, we observed significant variation depending on the description, \emph{F}(3,608) = 3.01, \emph{p} = .032, partial \(\eta\)\textsuperscript{2} = 0.001. When correcting for multiple comparisons, pairwise comparisons did not reveal significant differences between descriptions. We note that without correction, \emph{Francis} appeared to be rated as more moral than both \emph{Robin} (\emph{p} = .012), and \emph{Sam} (\emph{p} = .009). For the non-diagnostic descriptions there was no significant difference in ratings depending on description, \emph{t}(211) = -0.09, \emph{p} = .929, \emph{d} = 0.01.

We conducted a linear-mixed-effects model to test if condition influenced MPS-4 responses. Our outcome measure was MPS-4, our predictor variable was condition; we allowed intercepts and the effect of condition to vary across participants.
Overall, the model significantly predicted participants responses, and provided a better fit for the data than the baseline model, \(\chi\)\textsuperscript{2}(2) = 860.16, \emph{p} \textless{} .001. Condition was a significant predictor in the model \(b\) = -0.49, \emph{t}(211.05) = -8.54, \emph{p} \textless{} .001, with the non-diagnostic descriptions being rated as more moral than the diagnostic descriptions of immoral characters Figure~\ref{fig:pilot1bothconditionplot}.

\begin{figure}[!h]
\includegraphics[width=\textwidth,]{Supplementary_files/figure-latex/pilot1bothconditionplot-1} \caption{Pilot Study 1: Differences in moral perception depending on condition}\label{fig:pilot1bothconditionplot}
\end{figure}

We conducted a linear-mixed-effects model to test if condition influenced MM-1 responses. Our outcome measure was MM-1, our predictor variable was condition; we allowed intercepts and the effect of condition to vary across participants. Overall, the model significantly predicted participants responses, and provided a better fit for the data than the baseline model, \(\chi\)\textsuperscript{2}(2) = 924.82, \emph{p} \textless{} .001. Condition was a significant predictor in the model \(b\) = -8.22, \emph{t}(210.98) = -8.60, \emph{p} \textless{} .001, with the non-diagnostic descriptions being rated as more moral than the diagnostic descriptions, see Figure~\ref{fig:pilot1bothconditionplot}.

\hypertarget{pilot-study-2}{%
\section{Pilot Study 2}\label{pilot-study-2}}

Study 1 showed the moral dilution effect for judgments of \emph{bad} characters. The aim of this pilot study was to develop and test materials that may be used to study the moral dilution effect for judgments of morally \emph{good} characters. As with Pilot Study 1, we developed diagnostic and non-diagnostic descriptions. We hypothesized that evaluations of the diagnostic descriptions would be more extreme (more moral) than for the non-diagnostic descriptions

\hypertarget{pilot-study-2-method}{%
\subsection{Pilot Study 2: Method}\label{pilot-study-2-method}}

\hypertarget{pilot-2-participants-and-design}{%
\subsubsection{Pilot 2: Participants and design}\label{pilot-2-participants-and-design}}

The pilot study was a within-subjects design. The independent variable was description type with two levels, \emph{diagnostic} and \emph{non-diagnostic}. We used the same two dependent variables as in previous studies, the four item moral perception scale (MPS-4, \(\alpha\) = 0.84), and the single item moral perception measure (MM-1).

A total sample of 245 (70 female, 175 male, 0 non-binary, 0 prefer not to say; \emph{M}\textsubscript{age} = 36.69, min = 18, max = 71, \emph{SD} = 9.57) started the survey. Participants were recruited from MTurk.

We removed participants who failed both manipulation checks (\emph{n} = 30), leaving a total sample of 215 participants (63 female, 152 male, 0 non-binary, 0 prefer not to say; \emph{M}\textsubscript{age} = 36.59, min = 18, max = 71, \emph{SD} = 9.59).

\hypertarget{pilot-2-procedure-and-materials}{%
\subsubsection{Pilot 2: Procedure and materials}\label{pilot-2-procedure-and-materials}}

Data were collected using an online questionnaire presented with Qualtrics (www.qualtrics.com). Participants were presented with descriptions of six characters.

Moral character descriptions were developed by combining descriptions relating to three different moral foundations, focusing on upholding the moral foundations (rather than transgressions as in previous studies). We developed 4 descriptions of moral characters that read as follows: (i) \emph{Imagine a person named Sam. Throughout their life they have been known to always help and care for others, treat everyone fairly and equally, and show a strong sense of loyalty to others}; (ii) \emph{Imagine a person named Robin. Throughout their life they have been known to show compassion and empathy for others, act with a sense of fairness and justice, and, never to break their word}; (iii) \emph{Imagine a person named Francis. Throughout their life they have been known to uphold the standards of purity and decency, show respect for authority, and to always act honestly and fairly}; (iv) \emph{Imagine a person named Alex. Throughout their life they have been known to protect and provide shelter to the weak and vulnerable, uphold the rights of others, and show respect for authority}. We developed 2 descriptions of morally neutral characters that included information relating to physical appearance/attributes, hobbies/activities, and a color preference: (i) \emph{Imagine a person named Jackie. They have dark hair, go for a jog twice a week, and their favourite colour is blue}; (ii) \emph{Imagine a person named Charlie. They have blue eyes, drink coffee in the morning, and their favourite colour is green}.

We used the same gender ambiguous names, and we did not specify the gender of the characters. Pilot Study 2 was pre-registered at \color{blue}\url{https://aspredicted.org/W52_VPX}\color{black}.

\hypertarget{pilot-2-results}{%
\subsection{Pilot 2: Results}\label{pilot-2-results}}

The means and standard deviations for MPS-4 for each scenario are as follows:
\emph{Sam} (diagnostic),
\emph{M}\textsubscript{MPS-4} = 6.01, \emph{SD}\textsubscript{MPS-4} = 0.91,
\emph{Francis} (diagnostic),
\emph{M}\textsubscript{MPS-4} = 5.89, \emph{SD}\textsubscript{MPS-4} = 0.95,
\emph{Alex} (diagnostic),
\emph{M}\textsubscript{MPS-4} = 5.94, \emph{SD}\textsubscript{MPS-4} = 0.94,
\emph{Robin} (diagnostic),
\emph{M}\textsubscript{MPS-4} = 5.93, \emph{SD}\textsubscript{MPS-4} = 0.92,
\emph{Jackie} (non-diagnostic),
\emph{M}\textsubscript{MPS-4} = 5.60, \emph{SD}\textsubscript{MPS-4} = 0.99,
\emph{Charlie} (non-diagnostic),
\emph{M}\textsubscript{MPS-4} = 5.53, \emph{SD}\textsubscript{MPS-4} = 1.08. For the diagnostic descriptions, there was significant variation depending on the description, \emph{F}(3,613) = 2.91, \emph{p} = .036, partial \(\eta\)\textsuperscript{2} = 0.00, \emph{Sam} was viewed significantly more favorably than \emph{Francis} (\emph{p} = .040). For the non-diagnostic descriptions there was no significant difference in ratings depending on description, \emph{t}(214) = -1.79, \emph{p} = .075, \emph{d} = 0.12.

The means and standard deviations for MM-1 for each scenario are as follows:
\emph{Sam} (diagnostic),
\emph{M}\textsubscript{MM-1} = 79.85, \emph{SD}\textsubscript{MM-1} = 15.44;
\emph{Francis} (diagnostic),
\emph{M}\textsubscript{MM-1} = 78.30, \emph{SD}\textsubscript{MM-1} = 15.84;
\emph{Alex} (diagnostic),
\emph{M}\textsubscript{MM-1} = 79.78, \emph{SD}\textsubscript{MM-1} = 15.71;
\emph{Robin} (diagnostic),
\emph{M}\textsubscript{MM-1} = 79.46, \emph{SD}\textsubscript{MM-1} = 15.41;
\emph{Jackie} (non-diagnostic),
\emph{M}\textsubscript{MM-1} = 73.44, \emph{SD}\textsubscript{MM-1} = 15.83;
\emph{Charlie} (non-diagnostic),
\emph{M}\textsubscript{MM-1} = 73.07, \emph{SD}\textsubscript{MM-1} = 16.22. For the diagnostic descriptions, we observed no significant variation depending on the description, \emph{F}(3,594) = 1.45, \emph{p} = .231, partial \(\eta\)\textsuperscript{2} = 0.002. For the non-diagnostic descriptions there was no significant difference in ratings depending on description, \emph{t}(214) = -0.60, \emph{p} = .552, \emph{d} = 0.04.

\begin{figure}[!h]
\includegraphics[width=\textwidth,]{Supplementary_files/figure-latex/pilot2bothconditionplot-1} \caption{Pilot Study 2: Differences in moral perception depending on condition}\label{fig:pilot2bothconditionplot}
\end{figure}

We conducted a linear-mixed-effects model to test if condition influenced MPS-4 responses. Our outcome measure was MPS-4, our predictor variable was condition; we allowed intercepts and the effect of condition to vary across participants.
Overall, the model significantly predicted participants responses, and provided a better fit for the data than the baseline model, \(\chi\)\textsuperscript{2}(2) = 475.42, \emph{p} \textless{} .001. Condition was a significant predictor in the model \(b\) = 0.19, \emph{t}(214.35) = 6.53, \emph{p} \textless{} .001, with the diagnostic descriptions being rated as more moral than the non-diagnostic descriptions of immoral characters Figure~\ref{fig:pilot2bothconditionplot}.

We conducted a linear-mixed-effects model to test if condition influenced MM-1 responses. Our outcome measure was MM-1, our predictor variable was condition; we allowed intercepts and the effect of condition to vary across participants. Overall, the model significantly predicted participants responses, and provided a better fit for the data than the baseline model, \(\chi\)\textsuperscript{2}(2) = 324.13, \emph{p} \textless{} .001. Condition was a significant predictor in the model \(b\) = 3.04, \emph{t}(214.90) = 6.02, \emph{p} \textless{} .001, with the diagnostic descriptions being rated as more moral than the non-diagnostic descriptions, see Figure~\ref{fig:pilot2bothconditionplot}.

\newpage

\hypertarget{supplementary-materials}{%
\section{Supplementary Materials}\label{supplementary-materials}}

\hypertarget{descriptions-pilot-study-1-study-1}{%
\subsection{Descriptions (Pilot Study 1 \& Study 1)}\label{descriptions-pilot-study-1-study-1}}

\hypertarget{diagnostic-descriptions}{%
\subsubsection{Diagnostic Descriptions}\label{diagnostic-descriptions}}

Each moral description contains descriptive information relating to three different moral foundations as follows: \emph{Sam}: care, fairness, loyalty; \emph{Robin}: care, fairness, loyalty; \emph{Francis}: purity, authority, fairness; \emph{Alex}: care, fairness, authority.

\hypertarget{sam}{%
\subsubsection{Sam}\label{sam}}

Imagine a person named Sam.
Throughout their life they have been known to be cruel, act unfairly, and to betray their own group.

\hypertarget{robin}{%
\paragraph{Robin}\label{robin}}

Imagine a person named Robin.
Throughout their life they have been known to physically hurt others, treat some people differently to others, and show lack of loyalty.

\hypertarget{francis}{%
\subsubsection{Francis}\label{francis}}

Imagine a person named Francis.
Throughout their life they have been known to violate the standards of purity and decency, show lack of respect for authority, and treat people unequally.

\hypertarget{alex}{%
\subsubsection{Alex}\label{alex}}

Imagine a person named Alex.
Throughout their life they have been known to cause others to suffer emotionally, to deny others their rights, and to cause chaos or disorder.

\hypertarget{non-diagnostic-descriptions}{%
\subsubsection{Non-Diagnostic Descriptions}\label{non-diagnostic-descriptions}}

\hypertarget{jackie}{%
\subsubsection{Jackie}\label{jackie}}

Imagine a person named Jackie.
They have red hair, play tennis four times a month, and have one older sibling and one younger sibling.

\hypertarget{charlie}{%
\subsubsection{Charlie}\label{charlie}}

Imagine a person named Charlie.
They are left-handed, drink tea in the morning, and have two older siblings and one younger sibling.

\newpage

\hypertarget{descriptions-pilot-study-2-study-2-study-4}{%
\subsection{Descriptions (Pilot Study 2, Study 2 \& Study 4)}\label{descriptions-pilot-study-2-study-2-study-4}}

\hypertarget{diagnostic-descriptions-1}{%
\subsubsection{Diagnostic Descriptions}\label{diagnostic-descriptions-1}}

Each moral description contains descriptive information relating to three different moral foundations as follows: \emph{Sam}: care, fairness, loyalty; \emph{Robin}: care, fairness, loyalty; \emph{Francis}: purity, authority, fairness; \emph{Alex}: care, fairness, authority.

\hypertarget{sam-1}{%
\subsubsection{Sam}\label{sam-1}}

Imagine a person named Sam.
Throughout their life they have been known to always help and care for others, treat everyone fairly and equally, and show a strong sense of loyalty to others.

\hypertarget{robin-1}{%
\subsubsection{Robin}\label{robin-1}}

Imagine a person named Robin.
Throughout their life they have been known to show compassion and empathy for others, act with a sense of fairness and justice, and, never to break their word.

\hypertarget{francis-1}{%
\subsubsection{Francis}\label{francis-1}}

Imagine a person named Francis.
Throughout their life they have been known to uphold the standards of purity and decency, show respect for authority, and to always act honestly and fairly.

\hypertarget{alex-1}{%
\subsubsection{Alex}\label{alex-1}}

Imagine a person named Alex.
Throughout their life they have been known to protect and provide shelter to the weak and vulnerable, uphold the rights of others, and show respect for authority.

\hypertarget{non-diagnostic}{%
\subsection{Non-Diagnostic}\label{non-diagnostic}}

\hypertarget{jackie-1}{%
\subsubsection{Jackie}\label{jackie-1}}

Imagine a person named Jackie.
They have dark hair, go for a jog twice a week, and their favorite color is blue.

\hypertarget{charlie-1}{%
\subsubsection{Charlie}\label{charlie-1}}

Imagine a person named Charlie.
They have blue eyes, drink coffee in the morning, and their favorite color is green.

\newpage

\hypertarget{descriptions-study-3-study-5}{%
\subsection{Descriptions (Study 3 \& Study 5)}\label{descriptions-study-3-study-5}}

\hypertarget{diagnostic-descriptions-2}{%
\subsubsection{Diagnostic Descriptions}\label{diagnostic-descriptions-2}}

\hypertarget{sam-good}{%
\paragraph{Sam (good)}\label{sam-good}}

Imagine a person named Sam.
Throughout their life they have been known to always help and care for others, treat everyone fairly and equally, and show a strong sense of loyalty to others.

\hypertarget{robin-good}{%
\paragraph{Robin (good)}\label{robin-good}}

Imagine a person named Robin.
Throughout their life they have been known to show compassion and empathy for others, act with a sense of fairness and justice, and, never to break their word.

\hypertarget{alex-bad}{%
\paragraph{Alex (bad)}\label{alex-bad}}

Imagine a person named Alex.
Throughout their life they have been known to be cruel, act unfairly, and to betray their own group.

\hypertarget{francis-bad}{%
\paragraph{Francis (bad)}\label{francis-bad}}

Imagine a person named Francis.
Throughout their life they have been known to physically hurt others, treat some people differently to others, and show lack of loyalty.

\hypertarget{non-diagnostic-descriptions-1}{%
\subsubsection{Non Diagnostic Descriptions}\label{non-diagnostic-descriptions-1}}

They have red hair, play tennis four times a month, and have one older sibling and one younger sibling.

They are left-handed, drink tea in the morning, and have two older siblings and one younger sibling.

\newpage

\hypertarget{measures}{%
\subsection{Measures}\label{measures}}

\hypertarget{four-item-moral-perception-scale-mps-4}{%
\subsubsection{Four-item Moral Perception Scale (MPS-4)}\label{four-item-moral-perception-scale-mps-4}}

Please rate \_\_\_\_ along the following dimensions:

\begin{figure}
\centering
\includegraphics{../resources/images/mps4.png}
\caption{Screenshot of the MPS-4 items as presented to participants}
\end{figure}

\hypertarget{single-item-moral-perception-measure-mm-1}{%
\subsubsection{Single-item Moral Perception Measure (MM-1)}\label{single-item-moral-perception-measure-mm-1}}

Please rate \_\_\_\_ according to immoral or moral you view them:

\includegraphics{../resources/images/mm1.png}
\newpage

\newpage

\hypertarget{supplementary-analyses}{%
\section{Supplementary Analyses}\label{supplementary-analyses}}

\hypertarget{pilot-study-1-1}{%
\section{Pilot Study 1}\label{pilot-study-1-1}}

\hypertarget{pilot-1-differences-between-moral-descriptions}{%
\subsection{Pilot: 1: Differences Between Moral Descriptions}\label{pilot-1-differences-between-moral-descriptions}}

We developed a combined moral perception measure by calculating the mean of the combined mean-centered scores for MPS-4 and MM-1, and mean-centering this result. Below we report the analyses for this combined measure.

The standardized means and standard deviations for the combined measure for each scenario are as follows:
\emph{Sam} (diagnostic),
\emph{M} = -0.30, \emph{SD} = 1.16;
\emph{Francis} (diagnostic),
\emph{M} = -0.22, \emph{SD} = 1.06;
\emph{Alex} (diagnostic),
\emph{M} = -0.25, \emph{SD} = 1.10;
\emph{Robin} (diagnostic),
\emph{M} = -0.31, \emph{SD} = 1.19;
\emph{Jackie} (non-diagnostic),
\emph{M} = 0.36, \emph{SD} = 0.55;
\emph{Charlie} (non-diagnostic),
\emph{M} = 0.35, \emph{SD} = 0.55. For the moral descriptions, we observed significant variation depending on the description, \emph{F}(3,602) = 2.67, \emph{p} = .050, partial \(\eta\)\textsuperscript{2} = 0.001. When correcting for multiple comparisons, pairwise comparisons did not reveal significant differences between descriptions. We note that without correction, \emph{Francis} appeared to be rated as more moral than both \emph{Robin} (\emph{p} = .022), and \emph{Sam} (\emph{p} = .021). For the neutral descriptions there was no significant difference in ratings depending on description, \emph{t}(211) = -0.46, \emph{p} = .645, \emph{d} = 0.03.

\hypertarget{pilot-1-testing-moral-vs-neutral}{%
\subsection{Pilot 1: Testing Moral vs Neutral}\label{pilot-1-testing-moral-vs-neutral}}

We conducted a linear-mixed-effects model to test if condition influenced responses on this combined measure. Overall, the model significantly predicted participants responses, and provided a better fit for the data than the baseline model \(\chi\)\textsuperscript{2}(2) = 1,035.36, \emph{p} \textless{} .001, and condition was a significant predictor in the model \(b\) = -0.31, \emph{t}(210.99) = -8.74, \emph{p} \textless{} .001.

\begin{figure}[!h]
\includegraphics[width=\textwidth,]{Supplementary_files/figure-latex/pilot1cobminedconditionplot-1} \caption{Pilot Study 1: Differences in combined measure depending on condition}\label{fig:pilot1cobminedconditionplot}
\end{figure}

\newpage

\hypertarget{study-1}{%
\section{Study 1}\label{study-1}}

Again, we created a combined measure of moral perception from both DVs.

The means and standard deviations for the combined measure for each scenario are as follows:
\emph{Sam},
\emph{M} = 0.02, \emph{SD} = 0.89,
\emph{Francis},
\emph{M} = 0.48, \emph{SD} = 1.00,
\emph{Alex},
\emph{M} = -0.21, \emph{SD} = 0.92,
\emph{Robin},
\emph{M} = -0.32, \emph{SD} = 0.94. There was significant variation depending on the description, \emph{F}(3,2255) = 269.01, \emph{p} \textless{} .001, partial \(\eta\)\textsuperscript{2} = 0.10. \emph{Francis} appeared to be rated as the most favorable, followed by \emph{Sam}, then \emph{Alex} and finally \emph{Robin} as the least favorable (all \emph{p}s \textless{} .001).

We conducted a linear-mixed-effects model to test if condition influenced moral perception. Our outcome measure was the combined moral perception measure, our predictor variable was condition; we allowed intercepts and the effect of condition to vary across participants, and scenario was also included in the model.
Overall, the model significantly predicted participants responses, and provided a better fit for the data than the baseline model, \(\chi\)\textsuperscript{2}(8) = 762.31, \emph{p} \textless{} .001. Condition significantly influenced responses to the MPS-4, \emph{F}(1, 799.66) = 57.93, \emph{p} \textless{} .001; and was a significant predictor in the model when controlling for scenario, \(b\) = -0.08, \emph{t}(2,501.32) = -3.42, \emph{p} \textless{} .001, with the non-diagnostic descriptions being rated as more moral than the diagnostic (morally relevant) descriptions of immoral characters Figure~\ref{fig:S1combinedconditionplot}.

\begin{figure}[!h]
\includegraphics[width=\textwidth,]{Supplementary_files/figure-latex/S1combinedconditionplot-1} \caption{Study 1: Differences in combined measure depending on condition}\label{fig:S1combinedconditionplot}
\end{figure}

\newpage

\hypertarget{study-1-differences-between-the-descriptions}{%
\subsection{Study 1: Differences between the Descriptions}\label{study-1-differences-between-the-descriptions}}

We additionally conducted separate analyses for each scenario separately (for each dependent measure MPS-4, MM-1 and the combined measure). The responses for each scenario across each measure depending on condition are displayed in Figure~\ref{fig:S1allscenariosPlot}.

\begin{figure}[!p]
\includegraphics{Supplementary_files/figure-latex/S1allscenariosPlot-1} \caption{Study 1: Differences in moral perception for each description}\label{fig:S1allscenariosPlot}
\end{figure}

For \emph{Sam}, MPS-4 scores were significantly higher for the non-diagnostic condition (\emph{M} = 2.70, \emph{SD} = 0.82), than in the diagnostic condition (\emph{M} = 2.42, \emph{SD} = 0.87), \emph{t}(798.90) = -4.66, \emph{p} \textless{} .001, \emph{d} = 0.33; MM-1 ratings were higher in the non-diagnostic condition (\emph{M} = 26.55, \emph{SD} = 16.41), than in the diagnostic condition (\emph{M} = 21.50, \emph{SD} = 15.59), \emph{t}(787.84) = -4.45, \emph{p} \textless{} .001, \emph{d} = 0.32. For the combined measure ratings were also higher in the non-diagnostic condition (\emph{M} = 0.18, \emph{SD} = 0.88), than in the diagnostic condition (\emph{M} = -0.13, \emph{SD} = 0.88), \emph{t}(795.41) = -4.98, \emph{p} \textless{} .001, \emph{d} = 0.35.

For \emph{Robin}, MPS-4 scores were not significantly different for the non-diagnostic condition (\emph{M} = 2.16, \emph{SD} = 0.90), than in the diagnostic condition (\emph{M} = 2.09, \emph{SD} = 0.92), \emph{t}(793.94) = -1.09, \emph{p} = .275, \emph{d} = 0.08; MM-1 ratings were similar in the non-diagnostic condition (\emph{M} = 21.29, \emph{SD} = 16.94), and in the diagnostic condition (\emph{M} = 19.87, \emph{SD} = 17.17), \emph{t}(794.97) = -1.18, \emph{p} = .239, \emph{d} = 0.08. For the combined measure ratings were also similar in the non-diagnostic condition (\emph{M} = -0.28, \emph{SD} = 0.94), than in the diagnostic condition (\emph{M} = -0.36, \emph{SD} = 0.94), \emph{t}(796.03) = -1.24, \emph{p} = .217, \emph{d} = 0.09.

For \emph{Alex}, MPS-4 scores were significantly higher for the non-diagnostic condition (\emph{M} = 2.41, \emph{SD} = 0.88), than in the diagnostic condition (\emph{M} = 2.23, \emph{SD} = 0.86), \emph{t}(796.97) = -2.92, \emph{p} = .004, \emph{d} = 0.21; MM-1 ratings were higher in the non-diagnostic condition (\emph{M} = 21.93, \emph{SD} = 16.47), than in the diagnostic condition (\emph{M} = 19.20, \emph{SD} = 16.73), \emph{t}(798.89) = -2.33, \emph{p} = .020, \emph{d} = 0.16. For the combined measure ratings were also higher in the non-diagnostic condition (\emph{M} = -0.12, \emph{SD} = 0.92), than in the diagnostic condition (\emph{M} = -0.30, \emph{SD} = 0.92), \emph{t}(798.40) = -2.82, \emph{p} = .005, \emph{d} = 0.20.

For \emph{Francis}, MPS-4 scores were significantly higher for the non-diagnostic condition (\emph{M} = 3.12, \emph{SD} = 0.95), than in the diagnostic condition (\emph{M} = 2.98, \emph{SD} = 0.97), \emph{t}(796.12) = -1.99, \emph{p} = .047, \emph{d} = 0.14; MM-1 ratings were not significantly different in the non-diagnostic condition (\emph{M} = 30.38, \emph{SD} = 17.17), than in the diagnostic condition (\emph{M} = 29.84, \emph{SD} = 18.56), \emph{t}(788.61) = -0.43, \emph{p} = .668, \emph{d} = 0.03. For the combined measure ratings were also similar in the non-diagnostic condition (\emph{M} = 0.53, \emph{SD} = 0.98), and in the diagnostic condition (\emph{M} = 0.44, \emph{SD} = 1.02), \emph{t}(794.36) = -1.29, \emph{p} = .198, \emph{d} = 0.09.

\newpage

\hypertarget{pilot-study-2-1}{%
\section{Pilot Study 2}\label{pilot-study-2-1}}

\hypertarget{pilot-2-differences-between-moral-descriptions}{%
\subsection{Pilot: 2: Differences Between Moral Descriptions}\label{pilot-2-differences-between-moral-descriptions}}

As in previous studies, we developed a combined moral perception measure by calculating the mean of the combined mean-centered scores for MPS-4 and MM-1, and mean-centering this result. Below we report the analyses for this combined measure.

The standardized means and standard deviations for the combined measure for each scenario are as follows:
\emph{Sam} (moral),
\emph{M} = 0.21, \emph{SD} = 0.91;
\emph{Francis} (moral),
\emph{M} = 0.10, \emph{SD} = 0.96;
\emph{Alex} (moral),
\emph{M} = 0.18, \emph{SD} = 0.94;
\emph{Robin} (moral),
\emph{M} = 0.16, \emph{SD} = 0.93;
\emph{Jackie} (neutral),
\emph{M} = -0.24, \emph{SD} = 1.01;
\emph{Charlie} (neutral),
\emph{M} = -0.30, \emph{SD} = 1.07. For the moral descriptions, we observed significant variation depending on the description, \emph{F}(3,588) = 2.90, \emph{p} = .039, partial \(\eta\)\textsuperscript{2} = 0.002\emph{Sam} was viewed significantly more favorably than \emph{Francis} (\emph{p} = .045). For the neutral descriptions there was no significant difference in ratings depending on description, \emph{t}(214) = -1.46, \emph{p} = .147, \emph{d} = 0.10.

\newpage

\hypertarget{pilot-2-testing-moral-vs-neutral}{%
\subsection{Pilot 2: Testing Moral vs Neutral}\label{pilot-2-testing-moral-vs-neutral}}

Overall, the model significantly predicted participants responses, and provided a better fit for the data than the baseline model \(\chi\)\textsuperscript{2}(2) = 564.98, \emph{p} \textless{} .001, and condition was a significant predictor in the model \(b\) = 0.22, \emph{t}(214.32) = 6.60, \emph{p} \textless{} .001.

\begin{figure}[!h]
\includegraphics[width=\textwidth,]{Supplementary_files/figure-latex/pilot2cobminedconditionplot-1} \caption{Pilot Study 1: Differences in combined measure depending on condition}\label{fig:pilot2cobminedconditionplot}
\end{figure}

\newpage

\hypertarget{study-2}{%
\section{Study 2}\label{study-2}}

Below we report the results for the combined measure of moral perception from both DVs. We additionally report the effect of condition on responses to each description individually

The means and standard deviations for the combined measure for each scenario are as follows:
\emph{Sam},
\emph{M} = 0.07, \emph{SD} = 0.97,
\emph{Francis},
\emph{M} = -0.17, \emph{SD} = 1.06,
\emph{Alex},
\emph{M} = 0.09, \emph{SD} = 1.02,
\emph{Robin},
\emph{M} = 0.07, \emph{SD} = 0.96. There was significant variation depending on the description, \emph{F}(3,2335) = 48.01, \emph{p} \textless{} .001, partial \(\eta\)\textsuperscript{2} = 0.01. \emph{Francis} appeared to be rated as the less favorable than all other characters (all \emph{p}s \textless{} .001).

We conducted a linear-mixed-effects model to test if condition influenced moral perception. Our outcome measure was the combined moral perception measure, our predictor variable was condition; we allowed intercepts and the effect of condition to vary across participants, and scenario was also included in the model.
Overall, the model significantly predicted participants responses, and provided a better fit for the data than the baseline model, \(\chi\)\textsuperscript{2}(8) = 142.42, \emph{p} \textless{} .001. Condition did not influence moral perception, \emph{F}(1, 2,452.92) = 0.88, \emph{p} = .349; and was not a significant predictor in the model when controlling for scenario, \(b\) = -0.01, \emph{t}(2,613.53) = -0.42, \emph{p} = .673, see Figure~\ref{fig:S3combinedconditionplot}.

\begin{figure}[!h]
\includegraphics[width=\textwidth,]{Supplementary_files/figure-latex/S3combinedconditionplot-1} \caption{Study 2: Differences in combined measure depending on condition}\label{fig:S3combinedconditionplot}
\end{figure}

\newpage

\hypertarget{study-2-differences-between-the-descriptions}{%
\subsection{Study 2: Differences between the Descriptions}\label{study-2-differences-between-the-descriptions}}

Below we provide analyses of the effect of condition on responses to each scenario individually. The responses for each scenario across each measure depending on condition are displayed in Figure~\ref{fig:S2allscenariosPlot}.

For \emph{Sam}, MPS-4 scores were not significantly different in the non-diagnostic condition (\emph{M} = 6.17, \emph{SD} = 0.89), than in the diagnostic condition (\emph{M} = 6.05, \emph{SD} = 1.06), \emph{t}(680.49) = -1.71, \emph{p} = .088, \emph{d} = 0.12; MM-1 ratings were similar in the non-diagnostic condition (\emph{M} = 84.90, \emph{SD} = 14.26), than in the diagnostic condition (\emph{M} = 84.20, \emph{SD} = 14.76), \emph{t}(744.17) = -0.69, \emph{p} = .490, \emph{d} = 0.05. For the combined measure ratings were also similar in the non-diagnostic condition (\emph{M} = 0.11, \emph{SD} = 0.93), than in the diagnostic condition (\emph{M} = 0.02, \emph{SD} = 1.03), \emph{t}(717.94) = -1.33, \emph{p} = .183, \emph{d} = 0.10.

\newpage

\begin{figure}[!p]
\includegraphics{Supplementary_files/figure-latex/S2allscenariosPlot-1} \caption{Study 2: Differences in moral perception for each description}\label{fig:S2allscenariosPlot}
\end{figure}

For \emph{Robin}, MPS-4 scores were not significantly different for the non-diagnostic condition (\emph{M} = 6.08, \emph{SD} = 1.00), than in the diagnostic condition (\emph{M} = 6.13, \emph{SD} = 0.98), \emph{t}(784.04) = 0.73, \emph{p} = .463, \emph{d} = 0.05; MM-1 ratings were similar in the non-diagnostic condition (\emph{M} = 84.12, \emph{SD} = 14.37), and in the diagnostic condition (\emph{M} = 85.98, \emph{SD} = 13.32), \emph{t}(800.09) = 1.92, \emph{p} = .055, \emph{d} = 0.13. For the combined measure ratings were also similar in the non-diagnostic condition (\emph{M} = 0.03, \emph{SD} = 0.98), than in the diagnostic condition (\emph{M} = 0.13, \emph{SD} = 0.95), \emph{t}(788.76) = 1.46, \emph{p} = .145, \emph{d} = 0.10.

For \emph{Alex}, MPS-4 scores were not significantly different for the non-diagnostic condition (\emph{M} = 6.11, \emph{SD} = 1.00), than in the diagnostic condition (\emph{M} = 6.14, \emph{SD} = 0.99), \emph{t}(737.60) = 0.32, \emph{p} = .746, \emph{d} = 0.02; MM-1 ratings were similar in the non-diagnostic condition (\emph{M} = 85.28, \emph{SD} = 14.31), than in the diagnostic condition (\emph{M} = 84.83, \emph{SD} = 15.51), \emph{t}(776.47) = -0.43, \emph{p} = .668, \emph{d} = 0.03. For the combined measure ratings were also similar in the non-diagnostic condition (\emph{M} = 0.09, \emph{SD} = 0.98), than in the diagnostic condition (\emph{M} = 0.09, \emph{SD} = 1.04), \emph{t}(767.89) = -0.06, \emph{p} = .952, \emph{d} = 0.00.

For \emph{Francis}, MPS-4 scores were not significantly different for the non-diagnostic condition (\emph{M} = 5.82, \emph{SD} = 1.05), than in the diagnostic condition (\emph{M} = 5.90, \emph{SD} = 1.08), \emph{t}(794.94) = 1.06, \emph{p} = .290, \emph{d} = 0.07; MM-1 ratings were not significantly different in the non-diagnostic condition (\emph{M} = 81.74, \emph{SD} = 15.67), than in the diagnostic condition (\emph{M} = 82.31, \emph{SD} = 14.90), \emph{t}(771.23) = 0.54, \emph{p} = .591, \emph{d} = 0.04. For the combined measure ratings were also similar in the non-diagnostic condition (\emph{M} = -0.20, \emph{SD} = 1.08), and in the diagnostic condition (\emph{M} = -0.14, \emph{SD} = 1.04), \emph{t}(777.51) = 0.88, \emph{p} = .379, \emph{d} = 0.06.

\newpage

\newpage

\hypertarget{study-3}{%
\section{Study 3}\label{study-3}}

Below we report the results for the combined measure of moral perception from both DVs. We additionally report the effect of condition on responses to each description individually

The means and standard deviations for the combined measure for each scenario are as follows:
\emph{Sam},
\emph{M} = 0.45, \emph{SD} = 0.52,
\emph{Francis},
\emph{M} = -0.63, \emph{SD} = 1.19,
\emph{Alex},
\emph{M} = -0.66, \emph{SD} = 1.15,
\emph{Robin},
\emph{M} = 0.43, \emph{SD} = 0.52. There was significant variation depending on the description, \emph{F}(1,1027) = 473.77, \emph{p} \textless{} .001, partial \(\eta\)\textsuperscript{2} = 0.26. Both the \emph{good} characters (\emph{Robin} and \emph{Sam}) were rated significantly more favorably than both the \emph{bad} characters (\emph{Alex} and \emph{Francis}; all \emph{p}s \textless{} .001). There were no differences between \emph{Robin} and \emph{Sam} (\emph{good}: \emph{p} = .366) or between \emph{Alex} and \emph{Francis} (\emph{bad}; (\emph{p} = .648)).

We conducted a linear-mixed-effects model to test if our predictors influenced responses on the combined moral perception measure. Our outcome measure was the combined moral perception measure, our predictor variables were condition and valence; we allowed intercepts and the effects of condition and valence to vary across participants.
Overall, the model significantly predicted participants responses, and provided a better fit for the data than the baseline model,
\(\chi\)\textsuperscript{2}(5) = 4,467.15, \emph{p} \textless{} .001.
Condition significantly influenced responses to the combined moral perception measure,
\emph{F}(1, 873) = 16.65, \emph{p} \textless{} .001
and was a significant predictor in the model when controlling for scenario, \(b\) = -0.02, \emph{t}(873.00) = -4.08, \emph{p} \textless{} .001;
valence significantly predicted responses,
\emph{F}(1, 873) = 1,598.27, \emph{p} \textless{} .001;
and there was no significant condition \(\times\) valence interaction,
\emph{F}(1, 873) = 0.03, \emph{p} = .867, see Figure~\ref{fig:S3combinedplot}.

\begin{figure}[!h]
\includegraphics[width=\textwidth,]{Supplementary_files/figure-latex/S3combinedplot-1} \caption{Study 3: Differences in the combined measure depending on condition}\label{fig:S3combinedplot}
\end{figure}

\newpage

\hypertarget{study-3-differences-between-the-descriptions}{%
\subsection{Study 3: Differences between the descriptions}\label{study-3-differences-between-the-descriptions}}

Again, we conducted separate analyses to investigate of condition on responses to each scenario individually. The responses for each scenario across each measure depending on condition are displayed in Figure~\ref{fig:S3allscenariosPlot}.

For \emph{Sam}, MPS-4 scores were significantly lower in the non-diagnostic condition (\emph{M} = 5.81, \emph{SD} = 1.09), than in the diagnostic condition (\emph{M} = 5.98, \emph{SD} = 0.97), \emph{t}(859.15) = 2.46, \emph{p} = .014, \emph{d} = 0.17; Similarly, MM-1 ratings were significantly lower in the non-diagnostic condition (\emph{M} = 79.64, \emph{SD} = 15.68), than in the diagnostic condition (\emph{M} = 82.37, \emph{SD} = 14.67), \emph{t}(867.08) = 2.66, \emph{p} = .008, \emph{d} = 0.18. For the combined measure ratings were also lower in the non-diagnostic condition (\emph{M} = 0.39, \emph{SD} = 0.54), than in the diagnostic condition (\emph{M} = 0.50, \emph{SD} = 0.50), \emph{t}(863.14) = 2.85, \emph{p} = .004, \emph{d} = 0.19.

\newpage

For \emph{Robin}, MPS-4 scores were not significantly different for the non-diagnostic condition (\emph{M} = 5.88, \emph{SD} = 0.96), than in the diagnostic condition (\emph{M} = 5.83, \emph{SD} = 1.14), \emph{t}(844.53) = -0.77, \emph{p} = .440, \emph{d} = 0.05; MM-1 ratings were similar in the non-diagnostic condition (\emph{M} = 80.92, \emph{SD} = 15.27), and in the diagnostic condition (\emph{M} = 80.70, \emph{SD} = 15.07), \emph{t}(871.98) = -0.22, \emph{p} = .828, \emph{d} = 0.01. For the combined measure ratings were also similar in the non-diagnostic condition (\emph{M} = 0.44, \emph{SD} = 0.51), than in the diagnostic condition (\emph{M} = 0.42, \emph{SD} = 0.54), \emph{t}(867.63) = -0.57, \emph{p} = .569, \emph{d} = 0.04.

For \emph{Alex}, MPS-4 scores were not significantly different for the non-diagnostic condition (\emph{M} = 4.08, \emph{SD} = 1.96), than in the diagnostic condition (\emph{M} = 3.97, \emph{SD} = 2.11), \emph{t}(865.81) = -0.80, \emph{p} = .421, \emph{d} = 0.05; MM-1 ratings were similar in the non-diagnostic condition (\emph{M} = 52.19, \emph{SD} = 31.29), than in the diagnostic condition (\emph{M} = 49.58, \emph{SD} = 32.95), \emph{t}(868.76) = -1.20, \emph{p} = .230, \emph{d} = 0.08. For the combined measure ratings were also similar in the non-diagnostic condition (\emph{M} = -0.62, \emph{SD} = 1.11), than in the diagnostic condition (\emph{M} = -0.70, \emph{SD} = 1.19), \emph{t}(867.67) = -1.04, \emph{p} = .301, \emph{d} = 0.07.

For \emph{Francis}, MPS-4 scores were not significantly different for the non-diagnostic condition (\emph{M} = 4.08, \emph{SD} = 2.07), than in the diagnostic condition (\emph{M} = 4.07, \emph{SD} = 2.07), \emph{t}(871.94) = -0.09, \emph{p} = .928, \emph{d} = 0.01; MM-1 ratings were not significantly different in the non-diagnostic condition (\emph{M} = 51.56, \emph{SD} = 32.68), than in the diagnostic condition (\emph{M} = 51.42, \emph{SD} = 33.70), \emph{t}(871.59) = -0.06, \emph{p} = .952, \emph{d} = 0.00. For the combined measure ratings were also similar in the non-diagnostic condition (\emph{M} = -0.63, \emph{SD} = 1.18), and in the diagnostic condition (\emph{M} = -0.64, \emph{SD} = 1.20), \emph{t}(871.88) = -0.08, \emph{p} = .939, \emph{d} = 0.01.

\begin{figure}[!p]
\includegraphics{Supplementary_files/figure-latex/S3allscenariosPlot-1} \caption{Study 3: Differences in moral perception for each description}\label{fig:S3allscenariosPlot}
\end{figure}

\newpage

\hypertarget{study-4}{%
\section{Study 4}\label{study-4}}

Below we report the results for the combined measure of moral perception from both DVs. We additionally report the effect of condition on responses to each description individually

The means and standard deviations for the combined measure for each scenario are as follows:
\emph{Sam},
\emph{M} = 0.03, \emph{SD} = 1.02,
\emph{Francis},
\emph{M} = -0.03, \emph{SD} = 0.98,
\emph{Alex},
\emph{M} = 0.02, \emph{SD} = 1.04,
\emph{Robin},
\emph{M} = 0.04, \emph{SD} = 1.01. There was significant variation depending on the description, \emph{F}(3,2493) = 4.32, \emph{p} = .005, partial \(\eta\)\textsuperscript{2} = 0.00. \emph{Francis} appeared to be rated as the less favorable than all other characters (all \emph{p}s \textless{} .001).

We conducted a linear-mixed-effects model to test if condition influenced moral perception. Our outcome measure was the combined moral perception measure, our predictor variable was condition; we allowed intercepts and the effect of condition to vary across participants, and scenario was also included in the model.
Overall, the model significantly predicted participants responses, and provided a better fit for the data than the baseline model, \(\chi\)\textsuperscript{2}(8) = 42.42, \emph{p} \textless{} .001. Condition did not influence moral perception, \emph{F}(1, 865.01) = 5.31, \emph{p} = .021; and was not a significant predictor in the model when controlling for scenario, \(b\) = -0.01, \emph{t}(2,541.03) = -0.82, \emph{p} = .410, see Figure~\ref{fig:S3combinedconditionplot}.

\begin{figure}[!h]
\includegraphics[width=\textwidth,]{Supplementary_files/figure-latex/S4combinedconditionplot-1} \caption{Study 4: Differences in combined measure depending on condition}\label{fig:S4combinedconditionplot}
\end{figure}

\newpage

\hypertarget{study-4-differences-between-the-descriptions}{%
\subsection{Study 4: Differences between the Descriptions}\label{study-4-differences-between-the-descriptions}}

As in previous studies, we additionally conducted separate analyses to investigate of condition on responses to each scenario individually. The responses for each scenario across each measure depending on condition are displayed in Figure~\ref{fig:S4allscenariosPlot}.

\begin{figure}[!p]
\includegraphics{Supplementary_files/figure-latex/S4allscenariosPlot-1} \caption{Study 4: Differences in moral perception for each description}\label{fig:S4allscenariosPlot}
\end{figure}

For \emph{Sam} (good), MPS-4 scores were significantly lower in the non-diagnostic condition (\emph{M} = 5.89, \emph{SD} = 0.91), than in the diagnostic condition (\emph{M} = 6.02, \emph{SD} = 0.95), \emph{t}(810.53) = 1.97, \emph{p} = .049, \emph{d} = 0.14; MM-1 ratings were similar in the non-diagnostic condition (\emph{M} = 79.75, \emph{SD} = 14.62), than in the diagnostic condition (\emph{M} = 83.25, \emph{SD} = 13.30), \emph{t}(845.88) = 3.66, \emph{p} \textless{} .001, \emph{d} = 0.25. For the combined measure ratings were also similar in the non-diagnostic condition (\emph{M} = -0.06, \emph{SD} = 1.03), than in the diagnostic condition (\emph{M} = 0.15, \emph{SD} = 1.01), \emph{t}(829.20) = 3.07, \emph{p} = .002, \emph{d} = 0.21.

For \emph{Robin} (good), MPS-4 scores were significantly lower for the non-diagnostic condition (\emph{M} = 5.95, \emph{SD} = 0.93), than in the diagnostic condition (\emph{M} = 5.94, \emph{SD} = 0.95), \emph{t}(811.83) = -0.20, \emph{p} = .841, \emph{d} = 0.01; MM-1 ratings were lower in the non-diagnostic condition (\emph{M} = 81.62, \emph{SD} = 14.28), than in the diagnostic condition (\emph{M} = 81.64, \emph{SD} = 14.02), \emph{t}(824.54) = 0.02, \emph{p} = .982, \emph{d} = 0.00. For the combined measure ratings were also lower in the non-diagnostic condition (\emph{M} = 0.04, \emph{SD} = 1.03), than in the diagnostic condition (\emph{M} = 0.04, \emph{SD} = 0.99), \emph{t}(828.47) = -0.10, \emph{p} = .919, \emph{d} = 0.01.

For \emph{Alex}, MPS-4 scores were not significantly different for the non-diagnostic condition (\emph{M} = 5.97, \emph{SD} = 0.91), than in the diagnostic condition (\emph{M} = 5.91, \emph{SD} = 0.99), \emph{t}(845.29) = -0.91, \emph{p} = .362, \emph{d} = 0.06; MM-1 ratings were similar in the non-diagnostic condition (\emph{M} = 81.93, \emph{SD} = 13.38), than in the diagnostic condition (\emph{M} = 80.51, \emph{SD} = 15.21), \emph{t}(850.53) = -1.46, \emph{p} = .145, \emph{d} = 0.10. For the combined measure ratings were also similar in the non-diagnostic condition (\emph{M} = 0.07, \emph{SD} = 0.98), than in the diagnostic condition (\emph{M} = -0.02, \emph{SD} = 1.09), \emph{t}(847.27) = -1.30, \emph{p} = .192, \emph{d} = 0.09.

For \emph{Francis}, MPS-4 scores were not significantly different for the non-diagnostic condition (\emph{M} = 5.87, \emph{SD} = 0.95), than in the diagnostic condition (\emph{M} = 5.91, \emph{SD} = 0.87), \emph{t}(787.36) = 0.77, \emph{p} = .443, \emph{d} = 0.05; MM-1 ratings were not significantly different in the non-diagnostic condition (\emph{M} = 80.54, \emph{SD} = 14.38), than in the diagnostic condition (\emph{M} = 80.75, \emph{SD} = 13.99), \emph{t}(809.63) = 0.21, \emph{p} = .832, \emph{d} = 0.01. For the combined measure ratings were also similar in the non-diagnostic condition (\emph{M} = -0.05, \emph{SD} = 0.99), and in the diagnostic condition (\emph{M} = -0.01, \emph{SD} = 0.98), \emph{t}(814.30) = 0.55, \emph{p} = .581, \emph{d} = 0.04.

\newpage

\hypertarget{study-5}{%
\section{Study 5}\label{study-5}}

The means and standard deviations for the combined measure for each scenario are as follows:
\emph{Sam},
\emph{M} = 84.52, \emph{SD} = 15.49;
\emph{Francis},
\emph{M} = 44.51, \emph{SD} = 35.08;
\emph{Alex},
\emph{M} = 45.85, \emph{SD} = 34.36;
\emph{Robin},
\emph{M} = 85.15, \emph{SD} = 14.61. There was significant variation depending on the description, \emph{F}(3,1746) = 351.55, \emph{p} \textless{} .001, partial \(\eta\)\textsuperscript{2} = 0.38. Both the \emph{good} characters (\emph{Robin} and \emph{Sam}) were rated significantly more favorably than both the \emph{bad} characters (\emph{Alex} and \emph{Francis}; all \emph{p}s \textless{} .001). There were no differences between \emph{Robin} and \emph{Sam} (\emph{good}: \emph{p} = .963) or between \emph{Alex} and \emph{Francis} (\emph{bad}; (\emph{p} = .976)).

We conducted a 2 \(\times\) 2 between subjects ANOVA to test for an interaction between valence and condition.
Condition did not influence responses to the combined measure,
\emph{F}(1, 1746) = 8.42, \emph{p} = .004;
valence significantly predicted responses,
\emph{F}(1, 1746) = 964.98, \emph{p} \textless{} .001;
and there was no significant condition \(\times\) valence interaction,
\emph{F}(1, 1746) = 0.04, \emph{p} = .841.

For the \emph{bad} characters, there was no significant difference in responses to the combined measure between the diagnostic condition (\emph{M} = -0.81, \emph{SD} = 1.15) and the non-diagnostic condition (\emph{M} = -0.70, \emph{SD} = 1.07) depending on condition, \emph{t}(834.36) = -1.23, \emph{p} = .221, \emph{d} = 0.09.

For the \emph{good} characters, there was a significant difference in responses to the combined measure between the diagnostic condition (\emph{M} = 0.61, \emph{SD} = 0.41) and the non-diagnostic condition (\emph{M} = 0.49, \emph{SD} = 0.44) depending on condition, \emph{t}(886.55) = 3.16, \emph{p} = .002, \emph{d} = 0.28.

\hypertarget{study-5-differences-between-the-descriptions}{%
\subsection{Study 5: Differences between the Descriptions}\label{study-5-differences-between-the-descriptions}}

Again, we conducted separate analyses to investigate of condition on responses to each scenario individually. The responses for each scenario across each measure depending on condition are displayed in Figure~\ref{fig:S5allscenariosPlot}.

\begin{figure}[!p]
\includegraphics{Supplementary_files/figure-latex/S5allscenariosPlot-1} \caption{Study 5: Differences in moral perception for each description}\label{fig:S5allscenariosPlot}
\end{figure}

For \emph{Sam}, MPS-4 scores were not significantly different in the non-diagnostic condition (\emph{M} = 5.87, \emph{SD} = 0.95), than in the diagnostic condition (\emph{M} = 5.91, \emph{SD} = 0.87), \emph{t}(886.55) = 3.16, \emph{p} = .002, \emph{d} = 0.28; MM-1 ratings were similar in the non-diagnostic condition (\emph{M} = 80.54, \emph{SD} = 14.38), than in the diagnostic condition (\emph{M} = 80.75, \emph{SD} = 13.99), \emph{t}(809.63) = 0.21, \emph{p} = .832, \emph{d} = 0.01. For the combined measure ratings were also similar in the non-diagnostic condition (\emph{M} = -0.05, \emph{SD} = 0.99), than in the diagnostic condition (\emph{M} = -0.01, \emph{SD} = 0.98), \emph{t}(814.30) = 0.55, \emph{p} = .581, \emph{d} = 0.04.

For \emph{Robin}, MPS-4 scores were not significantly different for the non-diagnostic condition (\emph{M} = 6.11, \emph{SD} = 0.86), than in the diagnostic condition (\emph{M} = 6.28, \emph{SD} = 0.83), \emph{t}(448.03) = 2.09, \emph{p} = .037, \emph{d} = 0.20; MM-1 ratings were similar in the non-diagnostic condition (\emph{M} = 83.45, \emph{SD} = 14.86), and in the diagnostic condition (\emph{M} = 87.03, \emph{SD} = 14.14), \emph{t}(448.96) = 2.62, \emph{p} = .009, \emph{d} = 0.25. For the combined measure ratings were also similar in the non-diagnostic condition (\emph{M} = 0.51, \emph{SD} = 0.43), than in the diagnostic condition (\emph{M} = 0.62, \emph{SD} = 0.41), \emph{t}(448.56) = 2.62, \emph{p} = .009, \emph{d} = 0.25.

For \emph{Alex}, MPS-4 scores were not significantly different for the non-diagnostic condition (\emph{M} = 3.78, \emph{SD} = 2.02), than in the diagnostic condition (\emph{M} = 3.67, \emph{SD} = 2.15), \emph{t}(406.27) = -0.55, \emph{p} = .582, \emph{d} = 0.05; MM-1 ratings were similar in the non-diagnostic condition (\emph{M} = 46.75, \emph{SD} = 33.74), than in the diagnostic condition (\emph{M} = 44.80, \emph{SD} = 35.13), \emph{t}(409.65) = -0.58, \emph{p} = .560, \emph{d} = 0.06. For the combined measure ratings were also similar in the non-diagnostic condition (\emph{M} = -0.71, \emph{SD} = 1.06), than in the diagnostic condition (\emph{M} = -0.77, \emph{SD} = 1.13), \emph{t}(406.61) = -0.58, \emph{p} = .562, \emph{d} = 0.06.

For \emph{Francis}, MPS-4 scores were not significantly different for the non-diagnostic condition (\emph{M} = 3.84, \emph{SD} = 2.05), than in the diagnostic condition (\emph{M} = 3.60, \emph{SD} = 2.27), \emph{t}(424.52) = -1.17, \emph{p} = .243, \emph{d} = 0.11; MM-1 ratings were not significantly different in the non-diagnostic condition (\emph{M} = 46.97, \emph{SD} = 34.05), than in the diagnostic condition (\emph{M} = 42.03, \emph{SD} = 35.99), \emph{t}(428.22) = -1.47, \emph{p} = .143, \emph{d} = 0.14. For the combined measure ratings were also similar in the non-diagnostic condition (\emph{M} = -0.69, \emph{SD} = 1.08), and in the diagnostic condition (\emph{M} = -0.84, \emph{SD} = 1.18), \emph{t}(425.90) = -1.35, \emph{p} = .179, \emph{d} = 0.13.

\newpage

\newpage

\hypertarget{references}{%
\section*{References}\label{references}}
\addcontentsline{toc}{section}{References}

\hypertarget{refs}{}
\begin{CSLReferences}{1}{0}
\leavevmode\vadjust pre{\hypertarget{ref-grizzard_validating_2020}{}}%
Grizzard, M., Fitzgerald, K., Francemone, C. J., Ahn, C., Huang, J., Walton, J., \ldots{} Eden, A. (2020). Validating the extended character morality questionnaire. \emph{Media Psychology}, \emph{23}(1), 107--130. \url{https://doi.org/10.1080/15213269.2019.1572523}

\leavevmode\vadjust pre{\hypertarget{ref-walker_better_2021}{}}%
Walker, A. C., Turpin, M. H., Fugelsang, J. A., \& Białek, M. (2021). Better the two devils you know, than the one you don't: {Predictability} influences moral judgments of immoral actors. \emph{Journal of Experimental Social Psychology}, \emph{97}, 104220. \url{https://doi.org/10.1016/j.jesp.2021.104220}

\end{CSLReferences}


\end{document}
